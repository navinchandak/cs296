%% LyX 2.0.2 created this file.  For more info, see http://www.lyx.org/.
%% Do not edit unless you really know what you are doing.
\documentclass[english]{article}
\usepackage[T1]{fontenc}
\usepackage[latin9]{inputenc}
\usepackage{graphicx}
\usepackage{babel}
\usepackage{cite}
\begin{document}

\title{Physics behind the simulation:a cs296 group 05 report}


\author{Preetham sreenivas\\
 110050073\\
preethamsreenivas@gmail.com\\
\\
Navin chandak\\
110050047\\
navinchandak92@gmail.com\\
\\
Hardik kothari\\
110050029\\
khardik84@gmail.com}


\date{january 25,2013}

\maketitle

\section{introduction}

The purpose of this report is to learn the basic document making using
latex by explaining the concepts of physics involved in the top-level
blocks of the box2d simulation done for the cs296 course .


\section*{Physics behind the simulation}


\subsection*{the pulley-system}

\includegraphics{doc/pulley_system.eps}

the container shown in the left is initailly empty, and stays at the
top. As the container gets filled its mass increases and moves down.
let the mass on the left side be $m_{1}$and the mass on the right
be $m_{2}$.

and let the tension in the string be $T$ . Free body diagrams~\cite{fbd} on
both sides give

\begin{equation}
m_{1}g-T=m_{1}a
\end{equation}


\begin{equation}
T-m_{2}g=m_{2}a
\end{equation}
 eliminating $T$ from the equations gives 
\begin{equation}
a=\frac{(m_{1}-m_{2})g}{m_{1}+m_{2}}
\end{equation}


therefore the system moves with an acceleration $\frac{(m_{1}-m_{2})g}{m_{1}+m_{2}}$ 


\subsection*{the pendulum}

\includegraphics{doc/Pendulum.eps}

The pendulum is given an initial impulse to the right making it oscillate
about the fixed point.

At a particular instant ,if the angle made by the pendulum with the
vertical is $\theta$ $radians$ and the tension in the string be
$T$ $N$ and assuming $\theta$ to be small and the mass of the bob
to be $m$ $kg$ and $a$ m/s$^{2}$be the acceleration 

Then,

\[
-mg\sin\theta=ma
\]
 
\[
\Longrightarrow a=-g\sin\theta
\]


we know angular acceleration~\cite{pendulum} $\alpha$ equals $\frac{a}{l}$ and also
equals $\frac{d^{2}\theta}{dt^{2}}$

\[
\Longrightarrow l\frac{d^{2}\theta}{dt^{2}}+g\sin\theta=0
\]


\[
\Longrightarrow\frac{d^{2}\theta}{dt^{2}}+\frac{g\sin\theta}{l}=0
\]


this is the equation of motion of a simple pendulum.


\subsection*{see-saw}

\includegraphics{doc/lever.eps}

when the sphere hits the plank on the left side, suppose that it gives
an angular impulse of $J$ N-m about the fulcrum.

Since this is an impulsive force the effects of gravity can be ignored.
Applying conservation of angular momentum~\cite{conservation-amomentum} about the fulcrum before
and after collision

Let the initial angular momentum b be L$_{1}$kg-m$^{2}/s$ 

mass of the sphere be m

initial velocity of the sphere be u

final velocity of the sphere be v

final angular momentum be L$_{2}$kg-m$^{2}/s$ 

the moment of inertia of the system be $I$ kg-m$^{2}$

$l$ m be the length of the plank 

$\theta$ radians be the angle made by the plank with the ground before
collision.

\begin{equation}
L_{1}=mvr
\end{equation}


where
\begin{equation}
r=\frac{l}{2\cos\theta}
\end{equation}


Final angular momentum
\begin{equation}
L_{2}=I\omega^{2}-mvr
\end{equation}
from conservation of angular momentum,

\begin{equation}
L_{1}=L_{2}
\end{equation}
 
\begin{equation}
\Longrightarrow mur=I\omega^{2}-mvr
\end{equation}


\begin{equation}
\Longrightarrow\omega=\sqrt{\frac{mvr+mur}{I}}
\end{equation}
 

after collision the whole system rotates with this angular velocity
till it hits the ground, when the plank hits the ground on the other
side , the block on the right flies away with velocity $\frac{l\omega}{2}$
in a direction $\frac{\pi}{2}-\theta$ to the horizontal.


\section*{conclusion}

In this report, using latex , we explained the physics behind the
simulation of the top level box 2d blocks . The working of each block
is described using diagrams , relevant physical equations and derivations.



\bibliography{doc/mybib}{}
\bibliographystyle{plain}

\end{document}
